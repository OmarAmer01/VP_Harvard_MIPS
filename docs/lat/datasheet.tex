\documentclass[a4paper, 11pt]{report}
\usepackage[utf8]{inputenc}
\usepackage{blindtext}
\usepackage{tabularx}
\usepackage{multicol}
\usepackage{tgschola}
\usepackage{longtable}
\setlength{\columnsep}{1cm}

\title{Data-sheet}
\author{Omar Amer}
\date{August 2021}

\begin{document}

\maketitle


\section{General Description}
RISC-Like pipelined general purpose processor with dedicated graphics and sound units with 8 32-bit registers and 128 Mb of ram with integrated VGA support and ICSP, programmed using custom, feature rich assembly.
\subsection{Ports}
\begin{center}
    \begin{tabular}{|c|c|c|}
        \hline
        Name   & Width  & Description        \\
        \hline
        Input  & 32-bit & User input port    \\
        \hline
        Output & 32-bit & Output port        \\
        \hline
        CLK    & 1-bit  & Clock              \\
        \hline
        RST    & 1-bit  & Asynchronous reset \\
        \hline
    \end{tabular}
\end{center}
\subsection{Registers}
\begin{center}
    \begin{tabular}{|c|l|}
        \hline
        Register & Description          \\
        \hline
        AX       & Accumulator Register \\
        \hline
        BX       & Base Register        \\
        \hline
        CX       & Counter Register     \\
        \hline
        DX       & Data Register        \\
        \hline
        GX       & Graphics Register    \\
        \hline
        EX       & Extra Register       \\
        \hline
        EY       & Extra Register       \\
        \hline
        EZ       & Extra Register       \\
        \hline
    \end{tabular}
\end{center}
\pagebreak
\section{Instruction Set}
\begin{center}
    \begin{tabular}{|l|p{27em}|}
        \hline
        \textbf{Instruction} & \textbf{Description}                                                                \\
        \hline
        NOP                  & No operation                                                                        \\
        \hline
        MOV dst, src/imm     & Moves data from src/imm to dst                                                      \\
        \hline
        NOT reg              & Bit-wise NOT                                                                        \\
        \hline
        AND dst, src/imm     & Bit-wise AND, result stored in dst.                                                 \\
        \hline
        OR dst, src/imm      & Bit-wise OR, result stored in dst                                                   \\
        \hline
        XOR dst, src/imm     & Bit-wise XOR, result stored in dst                                                  \\
        \hline
        XNOR dst, src/imm    & Bit-wise XNOR, result stored in dst                                                 \\
        \hline
        NOR dst, src/imm     & Bit-wise NOR, result stored in dst                                                  \\
        \hline
        NAND dst, src/imm    & Bit-wise NAND, result stored in dst                                                 \\
        \hline
        SLL reg, imm         & Logical shift LEFT, shifts \textbf{\textit{reg}} content by \textbf{\textit{imm}}.  \\
        \hline
        SRL reg, imm         & Logical shift RIGHT, shifts \textbf{\textit{reg}} content by \textbf{\textit{imm}}. \\
        \hline
        INC reg              & Increment reg                                                                       \\
        \hline
        DEC reg              & Decrement reg                                                                       \\
        \hline
        OUT reg/imm          & Output lower 8-bits of reg on port                                                                  \\
        \hline
        IN reg               & Take input from port and place data into reg                                        \\
        \hline
        ADD dst, src/imm     & ADD src/imm + dst, result stored in dst                                             \\
        \hline
        SUB dst, src/imm     & SUB dst - src/imm, result stored in dst                                             \\
        \hline
        MUL dst, src/imm     & MUL src/imm * dst, result stored in EX:EY                                           \\
        \hline
        DIV dst, src         & DIV dst/src, result stored in dst                                                   \\
        \hline
        PUSH reg/imm         & Push reg/imm onto stack                                                             \\
        \hline
        POP reg              & Pops stack top into reg                                                             \\
        \hline
        CMP reg1, reg2       & Performs reg1 - reg2, without changing register values                              \\
        \hline
        JMP [addr]           & Unconditional jump to [addr]                                                        \\
        \hline
        JZ [addr]            & Jump if zero flag = 1 to [addr]                                                     \\
        \hline
        JNZ [addr]           & Jump if zero flag = 0 to [addr]                                                     \\
        \hline
        JEQ [addr]           & Jump if \emph{equal}, i.e: zero flag = 1 to [addr]                                  \\
        \hline
        JNEQ [addr]          & Jump if \emph{not equal}, i.e: zero flag = 0 to [addr]                              \\
        \hline
        JG [addr]            & Jump if \emph{greater than}, i.e: sign flag = 0 \& zero flag = 0 to [addr]          \\
        \hline
        JL [addr]            & Jump if \emph{less than}, i.e: sign flag = 1 zero flag = 0 to [addr]                \\
        \hline
        JGE [addr]           & Jump if \emph{greater than or equal}, i.e: sign flag = 0 or zero flag = 1 to [addr] \\
        \hline
        JLE [addr]           & Jump if \emph{less than or equal}, i.e: sign flag = 1 or zero flag = 1 to [addr]    \\
        \hline
        LD reg, [addr]       & Load data from memory [addr] to reg                                                 \\
        \hline
        ST reg, [addr]       & Store data to memory [addr] from reg                                                \\
        \hline
        GRA                  & ?????                                                                               \\
        \hline
        SND                  & ?????                                                                               \\
        \hline
    \end{tabular}
\end{center}
\subsection{Instruction Word}
\subsubsection{Type A Instructions (One / Zero Operand)}
By hex instruction, we mean the full range of possible operand combinations for the given op code. for example,
the op code \it{31} means \textbf{PUSH BX}, \it{30} means \textbf{PUSH AX}, and \it{3F} means
\textbf{PUSH EZ}

\begin{center}

    \begin{tabular}{|l|c|c|c|}
        \hline
        \textbf{Instruction} & \textbf{OP Code (binary)} & \textbf{reg} & \textbf{Hex Instruction} \\
        \hline
        NOP                  & 000000                    &              & 00:07                    \\
        \cline{1-2}
        \cline{4-4}
        NOT reg              & 000001                    &              & 08:0F                    \\
        \cline{1-2}
        \cline{4-4}
        INC reg              & 000010                    &              & 10:17                    \\
        \cline{1-2}
        \cline{4-4}
        DEC reg              & 000011                    & (000:111)    & 18:1F                    \\
        \cline{1-2}
        \cline{4-4}
        IN reg               & 000100                    &              & 20:27                    \\
        \cline{1-2}
        \cline{4-4}
        OUT reg              & 000101                    &              & 28:2F                    \\
        \cline{1-2}
        \cline{4-4}
        PUSH reg             & 000110                    &              & 30:37                    \\
        \cline{1-2}
        \cline{4-4}
        POP reg              & 000111                    &              & 38:3F                    \\
        \hline
    \end{tabular}
\end{center}

\subsubsection{Type B Instructions}

\begin{center}
    \begin{tabular}{|l|c|c|c|c|}
        \hline
        \textbf{Instruction} & \textbf{OP Code} & \textbf{Operand 1} & \textbf{Operand 2} & \textbf{Hex Instruction} \\
        \hline
        MOV dst, src         & 001000           & (000:111)          & (000:111)          & 200:23F                  \\
        \hline
        UNUSED               & 001001           &                    &                    &                          \\
        \hline
        CMP reg1, reg2       & 001010           & (000:111)          & (000:111)          & 280:2BF                  \\
        \hline
        AND dst, src         & 001011           & (000:111)          & (000:111)          & 2CO:2FF                  \\
        \hline
        OR dst, src          & 001100           & (000:111)          & (000:111)          & 300:33F                  \\
        \hline
        XOR dst, src         & 001101           & (000:111)          & (000:111)          & 340:37F                  \\
        \hline
        XNOR dst, src        & 001110           & (000:111)          & (000:111)          & 380:3BF                  \\
        \hline
        NOR dst, src         & 001111           & (000:111)          & (000:111)          & 3CO:3FF                  \\
        \hline
        NAND dst, src        & 010000           & (000:111)          & (000:111)          & 400:43F                  \\
        \hline
        ADD dst, src         & 010001           & (000:111)          & (000:111)          & 440:47F                  \\
        \hline
        SUB dst, src         & 010010           & (000:111)          & (000:111)          & 480:4BF                  \\
        \hline
        MUL dst, src         & 010011           & (000:111)          & (000:111)          & 4CO:4FF                  \\
        \hline
        DIV dst, src         & 010100           & (000:111)          & (000:111)          & 500:53F                  \\
        \hline
    \end{tabular}
\end{center}
\pagebreak
\subsubsection{Type C Instructions}
\begin{center}
    \begin{tabular}{|l|c|c|c|c|}
        \hline
        \textbf{Instruction} & \textbf{OP Code} & \textbf{Operand 1} & \textbf{Operand 2} & \textbf{Hex Instruction} \\
        \hline
        MOV reg, imm32       & 010101           & (000:111)          &                    &                          \\
        \cline{1-3}
        \cline{5-5}
        CMP reg, imm32       & 010110           & (000:111)          &                    &                          \\
        \cline{1-3}
        \cline{5-5}
        AND reg, imm32       & 010111           & (000:111)          &                    &                          \\
        \cline{1-3}
        \cline{5-5}
        OR reg, imm32        & 011000           & (000:111)          &                    &                          \\
        \cline{1-3}
        \cline{5-5}
        XOR reg, imm32       & 011001           & (000:111)          & 32-bit immediate   &                          \\
        \cline{1-3}
        \cline{5-5}
        XNOR reg, imm32      & 011010           & (000:111)          &                    &                          \\
        \cline{1-3}
        \cline{5-5}
        NOR reg, imm32       & 011011           & (000:111)          &                    &                          \\
        \cline{1-3}
        \cline{5-5}
        NAND reg, imm32      & 011100           & (000:111)          &                    &                          \\
        \hline
        SLL reg, imm5        & 011101           & (000:111)          & 5-bit immediate    &                          \\
        \cline{1-3}
        \cline{5-5}
        SRL reg, imm5        & 011110           & (000:111)          &                    &                          \\
        \hline
        ADD reg, imm32       & 011111           & (000:111)          &                    &                          \\
        \cline{1-3}
        \cline{5-5}
        SUB reg, imm32       & 100000           & (000:111)          & 32-bit immediate   &                          \\
        \cline{1-3}
        \cline{5-5}
        MUL reg, imm32       & 100001           & (000:111)          &                    &                          \\
        \cline{1-3}
        \cline{5-5}
        DIV reg, imm32       & 100010           & (000:111)          &                    &                          \\
        \hline
        LD reg, imm11        & 110000           & (000:111)          & 11-bit immediate   &                          \\
        \hline
        ST reg, imm11        & 110001           & (000:111)          & 11-bit immediate   &                          \\
        \hline
    \end{tabular}
\end{center}

\subsubsection{Type D Instructions}
\begin{center}
    \begin{tabular}{|l|c|c|}
        \hline
        \textbf{Instruction} & \textbf{OP Code} & \textbf{Operand 1} \\
        \hline
        OUT imm32            & 100011           &                    \\
        \cline{1-2}
        PUSH imm32           & 100100           & 32-bit immediate   \\
        \hline
        JMP imm16            & 100101           &                    \\
        \cline{1-2}
        JZ imm16             & 100110           &                    \\
        \cline{1-2}
        JNZ imm16            & 100111           &                    \\
        \cline{1-2}
        JEQ imm16            & 101000           &                    \\
        \cline{1-2}
        JNEQ imm16           & 101001           & 16-bit immediate   \\
        \cline{1-2}
        JG imm16             & 101010           &                    \\
        \cline{1-2}
        JL imm16             & 101011           &                    \\
        \cline{1-2}
        JGE imm16            & 101100           &                    \\
        \cline{1-2}
        JLE imm16            & 101101           &                    \\
        \cline{1-2}
        \hline
    \end{tabular}
\end{center}

\subsubsection{Type X Instructions}
\begin{center}
    \begin{tabular}{|c|c|}
        \hline
        \textbf{Instruction} & \textbf{OP Code} \\
        \hline
        GRA                  & 101110           \\
        \hline
        SND                  & 101111           \\
        \hline
    \end{tabular}
\end{center}

\subsubsection{Instruction Size}
\begin{center}
    \begin{tabular}{|c|c|c|c|c|}
        \hline
        \textbf{Type} & \textbf{OP Code} & \textbf{Operand 1} & \textbf{Operand 2} \\
        \hline
        Type A        &                  & 3 bits             & ---                \\
        \cline{1-1}
        \cline{3-4}
        Type B        &                  & 3 bits             & 3 bits             \\
        \cline{1-1}
        \cline{3-4}
        Type C        & 6 bits           & 3 bits             & 32 bits            \\
        \cline{1-1}
        \cline{3-4}
        Type D        &                  & 32 bits            & ---                \\
        \cline{1-1}
        \cline{3-4}
        Type X        &                  &                    &                    \\
        \hline
    \end{tabular}
\end{center}

\section{Language Rules}
\paragraph{The VP uses a custom variant of assembly language called chasm. its rules are
shown below:}
\begin{itemize}
    \item Chasm is case-insensitve.
    \item Each line of code consists of a either:
     \begin{itemize}
        \item a mnemonic
        \item a mnemonic followed by a single operand
        \item a mnemonic followed by an operand, a colon "," ,and another operand.
    \end{itemize}
    \item The first operand is always a register.
    \item All numbers must begin with a 0?, where ? corresponds to the base of the number,
    for example, 0x53A is hex 53A, 0d123 is decimal 123, and 0b11011 is binary 11011.
    \item An instruction must have one space after the mnemonic, and one space after the colon (if applicable)
\end{itemize}

\end{document}

